\documentclass[12pt, a4paper, oneside]{ctexart}
\usepackage{subcaption,listings,amsmath, amsthm, amssymb, bm, color, framed, graphicx, hyperref, mathrsfs,minipage-marginpar}

\title{\textbf{数据库系统作业一}}
\author{2021113140符世博}
\date{}
\linespread{1.5}
\definecolor{shadecolor}{RGB}{241, 241, 255}
\newcounter{problemname}
\newenvironment{problem}{\begin{shaded}\stepcounter{problemname}\par\noindent\textbf{题目\arabic{problemname}. }}{\end{shaded}\par}
\newenvironment{solution}{\par\noindent\textbf{解答. }}{\par}
\newenvironment{note}{\par\noindent\textbf{题目\arabic{problemname}的注记. }}{\par}


\begin{document}
    \maketitle
    
    \begin{problem}
        在图书管理数据库中,有如下三个关系:

        图书信息关系:B(B\#, BNAME, AUTHOR, TYPE),其中B\#为图书编号,BNAME为书名,
        AUTHOR为作者,TYPE为类别;

        学生信息关系:S(S\#, SNAME, CLASS),其中S\#为学号,SNAME为学生姓名,CLASS为班级
        号;

        借阅信息关系:L(S\#, B\#, DATE),其中S\#为借阅人学号,B\#为被借阅图书编号,DATE为借阅日
        期。

        使用关系代数回答以下问题:

        (1)查询借阅了“《西游记》”这本书的学生的班级

        (2)查询“201”班学生借阅图书的书名

        (3)查询“小明”借过,但“小李”没有借过的图书的编号

        (4)查询借阅过“《红楼梦》”这本书的总学生数
    \end{problem}
    \begin{solution}
        \par
        (1)$\pi_{CLASS}(\sigma_{BNAME="\text{西游记}"}(B\bowtie S\bowtie L))$

        (2)$\pi_{BNAME}(\sigma_{CLASS="\text{201}"}(B\bowtie S\bowtie L))$

        (3)$\pi_{B\#}(\sigma_{SNAME="\text{小明}"}(S\bowtie L))-\pi_{B\#}(\sigma_{SNAME="\text{小李}"}(S\bowtie L))$

        (4)$|\sigma_{BNAME="\text{红楼梦}"}(B\bowtie L)|$
    \end{solution}

    \begin{problem}
        在学生成绩数据库中,有如下三个关系:

        学生信息关系:S(S\#, SNAME, D\#),其中S\#为学号,SNAME为学生姓名,D\#为所在系名;

        学生成绩关系:SC(S\#, C\#, Grade),其中S\#为学号,C\#为课程号,Grade为成绩;

        系信息关系:D(D\#, Addr),其中D\#为系名,Addr为所在地址

        使用关系代数回答以下问题:

        (1)查询“物理系”的全体学生

        (2)查询“化学系”的全体学生的学号和姓名

        (3)查询选修了“1002”课程但没有选修“1005”课程的学生

        (4)查询既选修了“1002”课程的学生中选修了“1003”课程的学生姓名
    \end{problem}
    \begin{solution}
        \par
        (1)$\sigma_{D\#="\text{物理系}"(S)}$

        (2)$\pi_{S\#,SNAME}(\sigma_{D\#="\text{化学系}"(S)})$

        (3)$\pi_{S\#}(\sigma_{C\#=1002}(SC))-\pi_{S\#}(\sigma_{C\#=1005}(SC))$

        (4)$\pi_{S\#}(\sigma_{C\#=1002}(SC\bowtie S))\cap\pi_{S\#}(\sigma_{C\#=1003}(SC\bowtie S))$

    \end{solution}

    \begin{problem}
        在第2题学生成绩数据库中,若S关系中没有空值,SC关系中Grade值可能为空值,则使用关系代数回答以下问题:

        (1)查询选过课的学生的学号和姓名

        (2)查询没选过课的学生的学号和姓名
    \end{problem}
    \begin{solution}
        \par
        (1)$\pi_{S\#,SNAME}(S\bowtie SC)$

        (2)$\pi_{S\#,SNAME}(S)-\pi_{S\#,SNAME}(S\bowtie SC)$
    \end{solution}

\end{document}